
\begin{tcolorbox}[colback=gray!5!white,colframe=gray!75!black,title=\textbf{Indirect Approach}]
\begin{itemize}
	\item \textit{first optimize then discretize}
	\item controls must be eliminatable
	\item controls might be discontinuous
	\item differential equation might mbe unstable, very non lin and not suitable for forward simulation\\
\end{itemize}
\textbf{Two Point Boundary Value Problem (TPBVP)}
\begin{align*}
	x^*(0) &= \bar{x}_0 & & \text{(init val)}\\
	\dot{x}^*(t) &= f(x^*(t), u^*(t)) \quad &t \in [0, T] \quad & \text{(ODE)}\\
	\dot{\lambda}^*(t) &= - \nabla_x H(x^*(t), \lambda^*(t), u^*(t)) \quad & t \in [0, T] \quad & \text{(adjoint eq.)}\\
	u^*(t) &= \arg \min_u H(x^*(t), \lambda^*(t), u) & & \text{(min.principle)}\\
	\lambda^*(T) &= \nabla E(x^*(T)) & & \text{(adjoint final val)}
\end{align*}
Equations 2,3,4 can be combined, such that
\begin{align*}
	y(t) = \begin{bmatrix}
	x(t) \\ \lambda(t)
	\end{bmatrix}
	 \quad
	 \dot{y} =
	 \begin{bmatrix}
		 f(x^*(t), u^*(t))\\
		 - \nabla_x H(x^*(t), \lambda^*(t), u^*(t))
	 \end{bmatrix}
 	\quad
 	y(0) = \begin{bmatrix}
		x_0\\ \lambda_0
 	\end{bmatrix}
\end{align*}
The only unkown is $\lambda_0$. Use newton-type method to find it s.t. $\lambda^*(T) = \nabla E(x^*(T)) $
\end{tcolorbox}