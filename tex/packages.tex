% math stuff
\usepackage[nosumlimits, nointlimits, fleqn]{amsmath}
\usepackage{amssymb}

% Bildumgebungen
\usepackage{graphicx}
\usepackage{float}
\usepackage{xcolor}	

% Tabellenumgebung
\usepackage{longtable}				% Tabellen mit Seitenumbruch
\renewcommand\arraystretch{1.4}   		% Tabellen: Zeilen um Faktor x vergrößern
\usepackage{booktabs}				% professionelle Tabellen (Grundeinstellung bei Excel2LaTeX Excel-PlugIn)

% Sonstige Umgegebungen
\usepackage{listings}	
\usepackage[ngerman]{babel}
\usepackage{iftex}
% \usepackage[latin1]{inputenc}
\usepackage[T1]{fontenc}

% Drei Spalten nebeneinander
\usepackage{multicol}

% Colorboxen
\usepackage{tcolorbox}

% Package um Aufz�hlungen zu "versch�nern"
\usepackage{enumitem}

% Todos hervorheben
\usepackage[obeyDraft]{todonotes}

% Um Einheiten richtig darzustellen 
\usepackage[per-mode=symbol]{siunitx}

% Für eine Verlinkung im Inhaltsverzeichnis.
% Dieses Paket möglichst als letztes einbinden falls Fehler auftreten!!!
\usepackage{hyperref}
\hypersetup{
    %bookmarks=true,
    unicode=false,
    pdfborder={0 0 0},
    pdftoolbar=true,
    pdfmenubar=true,
    pdffitwindow=false,
    pdfstartview={FitH},
    pdftitle={HelpSheet},
    pdfnewwindow=true,
    colorlinks=false,
    linkcolor=red,
    citecolor=green,
    filecolor=magenta,
    urlcolor=cyan
}

\usepackage{blindtext}
\usepackage{empheq}

\RequireXeTeX

%%% Local Variables:
%%% mode: xetex
%%% TeX-master: "../HelpSheet"
%%% End:
